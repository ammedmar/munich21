%!TEX root = ../munich21.tex

\begin{frame}{Viewpoint}
	\vskip -10pt
	\begin{block}{A primary goal of algebraic topology}
		To construct invariants of topological space up to some notion of equivalence by recasting into combinatorial and algebraic models.
	\end{block}

	\medskip \pause
	\begin{block}{A basic tension}
		Computability vs. strength of invariants.
	\end{block}

	\medskip \textcolor{pblue}{Example:}
	Homology vs. homotopy.

	\medskip \pause
	\begin{block}{A more subtle tension}
		Effectiveness vs. functoriality of their constructions.
	\end{block}

	\medskip \textcolor{pblue}{Example:}
	cohomology via a cochain complex or \\
	\hspace*{40pt} via maps to Eilenberg-Maclane spaces.
\end{frame}

\begin{frame}{Modeling spaces combinatorially}
	\pause
	\begin{block}{Poincar\'{e}}
		Break spaces into contractible combinatorial pieces: simplices, cubes, ...
	\end{block}

	\pause \textcolor{pblue}{Cohomology:}
	via a cochain complex generated by these pieces.

	\medskip \pause	More generally:
	\begin{block}{Kan-Quillen}
		Use category theory to replace spaces by functors with a geometric realization: simplicial sets, cubical sets, ...
	\end{block}

	\pause \textcolor{pblue}{Basic objects:}
	Chains on standard pieces $\gchains(\gsimplex^n)$, $\gchains(\gcube^n)$, ...

	\smallskip \pause
	\begin{block}{Our goal (loosly stated)}
		Understand these chain complexes deeply to enhance (co)homology with finer effectively computable invariants.
	\end{block}
\end{frame}

\begin{frame}{Shortcomings of (co)homology}
	\pause With mod 2 coefficients the Torus and the Klein bottle are isomorphic
	\[
	H^\bullet(\mathsf T; \Ftwo) \cong H^\bullet(\mathsf K; \Ftwo)
	\]
	as graded vector spaces.

	\bigskip \pause
	Similarly,
	\[
	H^\bullet(\mathbb{C} P^2; \Z) \cong H^\bullet(S^2 \vee S^4; \Z)
	\]
	as graded abelian groups.

	\bigskip \pause
	\begin{block}{Cup product}
		These can be distinguished by the algebra/ring structure in cohomology.
	\end{block}
\end{frame}

\begin{frame}{Intersection intuition}
	\pause Cup product corresponds to intersection via Poincar\'e duality
	\pause
	\begin{figure}
		\newcommand*{\xMin}{0}%
\newcommand*{\xMax}{4}%
\newcommand*{\yMin}{0}%
\newcommand*{\yMax}{4}%
\begin{subfigure}{.4\textwidth}
	\centering
	\begin{tikzpicture}[scale=.8]
	\draw[-{Latex[length=2mm]}] (-.5,\yMin)--(-.5,\yMax);
	\draw[-{Latex[length=2mm]}] (-.5,\yMin)--(-.5,\yMax-.5);
	\draw[-{Latex[length=2mm]}] (4.5,\yMin)--(4.5,\yMax);
	\draw[-{Latex[length=2mm]}] (4.5,\yMin)--(4.5,\yMax-.5);
	
	\draw[-{Latex[length=2mm]}] (\xMin, -.5)--(\xMax, -.5);
	\draw[-{Latex[length=2mm]}] (\xMin, 4.5)--(\xMax, 4.5);
	
	\draw (0,0)--(0,4)--(4,4)--(4,0)--(0,0);

	\draw[color=blue!50, very thick] (0,2) .. controls (1,2.5) and (3,1.5) .. (4,2);
	\draw[color=red!50, very thick] (0,1) .. controls (1,1.3) and (3,1) .. (4,1);
	\end{tikzpicture}
	\caption{Torus}
\end{subfigure}
\quad  
\begin{subfigure}{.4\textwidth}
	\centering
	\begin{tikzpicture}[scale=.8]
	\draw[-{Latex[length=2mm]}] (-.5,\yMin)--(-.5,\yMax);
	\draw[-{Latex[length=2mm]}] (-.5,\yMin)--(-.5,\yMax-.5);
	\draw[-{Latex[length=2mm]}] (4.5,\yMax)--(4.5,\yMin);
	\draw[-{Latex[length=2mm]}] (4.5,\yMax)--(4.5,\yMin+.5);
	
	\draw[-{Latex[length=2mm]}] (\xMin, -.5)--(\xMax, -.5);
	\draw[-{Latex[length=2mm]}] (\xMin, 4.5)--(\xMax, 4.5);
	
	\draw (0,0)--(0,4)--(4,4)--(4,0)--(0,0);
		
	\draw[color=blue!50, very thick] (0,2) .. controls (1,2.5) and (3,1.5) .. (4,2);
	\draw[color=red!50, very thick] (0,1) .. controls (1,1) and (2,2.5) .. (4,3);
	\end{tikzpicture}
	\caption{Klein Bottle}
\end{subfigure}
		\label{f:torus and klein bottle}
	\end{figure}

	\textbf{Torus}: the (transverse) self-intersection for any 1-cycle is always \textcolor{pblue}{even}.

	\medskip

	\textbf{Klein Bottle}: the self-intersection of the depicted 1-cycle is \textcolor{pblue}{odd}.
\end{frame}