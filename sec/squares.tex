%!TEX root = ../munich21.tex

\begin{frame}[fragile]{Cup product}
	\pause Alexander and Whitney defined the cup product by dualizing a chain approximation to the diagonal:
	\[
	\begin{split}
	\gchains(\gsimplex^n) &\to
	\gchains(\gsimplex^n) \otimes \gchains(\gsimplex^n) \\
	[v_0, \dots, v_m] &\mapsto
	\textstyle\sum_{i=1}^m \, [v_i, \dots, v_m] \otimes [v_i, \dots, v_m].
	\end{split}
	\]
	\pause Similarly, Cartan and Serre constructed: $\gchains(\gcube^n) \to \gchains(\gcube^n) \otimes \gchains(\gcube^n)$.

	\bigskip \pause
	As mentioned before, as graded rings,
	\[
	H^\bullet(\mathbb{C} P^2) \not\cong H^\bullet(S^2 \vee S^4).
	\]

	\vskip -8pt \pause But,
	\[
	H^\bullet(\Sigma(\mathbb{C} P^2)) \cong H^\bullet(\Sigma(S^2 \vee S^4)),
	\]
	where $\Sigma$ denotes suspension, for example $\Sigma(S^1)$ is
	\begin{center}
		\includegraphics[scale=.2]{aux/suspension.pdf}
	\end{center}
\end{frame}

\begin{frame}{Steenrod squares}
	\pause These chain approximations, unlike the diagonal of spaces, are \textcolor{pblue}{not} invariant under transposition: $x \otimes y \stackrel{T}{\mapsto} y \otimes x$.
	\begin{center}
		\begin{tikzpicture}
		\draw[color=pblue, thick] (0,0)--(1,1);
		\draw[->] (1.25, .5) -- (1.75, .5);
		\end{tikzpicture}
		\begin{tikzpicture}
		\node at (-0.1, 1){};
		\draw[color=pblue, thick] (0,0)--(1,0)--(1,1);
		\draw (1,1)--(0,1)--(0,0);
		\end{tikzpicture}
	\end{center}

	\medskip \pause To correct homotopically the breaking of this symmetry, Steenrod introduced explicit maps
	\[
	\Delta_i \colon \gchains(\gsimplex^n) \to \gchains(\gsimplex^n)^{\otimes 2},
	\qquad
	\partial \Delta_{i} = \big(1 \pm T \big) \Delta_{i-1}
	\]
	inducing further structure on mod 2 cohomology:
	\[
	\begin{split}
	Sq^k \colon H^\bullet(X; \Ftwo) &\to H^\bullet(X; \Ftwo) \\
	[\alpha] &\mapsto [(\alpha \otimes \alpha) \Delta_{k-|\alpha|}(-)]
	\end{split}
	\]
	\vskip -10pt \pause \textcolor{pblue}{Distinguishes}
	\[
	H^\bullet(\Sigma(\mathbb{C} P^2)) \not\cong H^\bullet(\Sigma(S^2 \vee S^4)).
	\]
\end{frame}

\begin{frame}[fragile]{A (new) description of Steenrod's construction}
	\pause \vskip -5pt \textcolor{pblue}{Notation:} \vspace*{-5pt}
	\[
	d_u[v_0, \dots, v_m] = [v_0, \dots, \widehat v_u, \dots, v_m]
	\]
	\pause \vspace*{-15pt}
	\[
	\rP_q(n) = \big\{ U \subseteq \{0,\dots,n\} : \bars{U} = q \big\}
	\]
	\pause \vspace*{-15pt}
	\[
	\forall \, U = \{u_1 < \dots < u_q\} \in \rP_q(n)
	\]
	\pause \vspace*{-15pt}
	\[
	d_U = d_{u_1} \dotsm \, d_{u_q}
	\]
	\pause \vspace*{-15pt}
	\[
	U^\varepsilon = \big\{ u_i \in U \mid u_i + i \equiv \varepsilon \text{ mod } 2 \big\}
	\]
	\pause \vskip -10pt
	\begin{definition}[Med.]
		For a basis element $x \in \gchains_m(\gsimplex^n)$
		\vspace*{-5pt}
		\[
		\Delta_i(x) \ = \!\!\! \sum_{U \in \rP_{m-i}(n)} \!\! d_{U^0}(x) \otimes d_{U^1}(x)
		\]
		\vspace*{-10pt}
	\end{definition}
	\pause \textcolor{pblue}{Example:} \vspace*{-5pt}
	\begin{align*}
	\Delta_0 [0,1,2] &=
	\Big( d_{12} \otimes \id + d_2 \otimes d_0 + \id \otimes d_{01} \Big) [0,1,2]^{\otimes 2} \\ &=
	[0] \otimes [0,1,2] + [0,1] \otimes [1,2] + [0,1,2] \otimes [2].
	\end{align*}
\end{frame}

\begin{frame}{Fast computation of Steenrod squares}
	\pause
	Comparing with SAGE: (algorithm based on EZ-AW contraction)

	\smallskip \pause
	\textcolor{pblue}{$Sq^1$} on \textcolor{pblue}{$\Sigma^i\R P^2$} ($i^\th$ suspension of the real projective plane)

	\medskip
	\includegraphics[width=\textwidth]{aux/comp_sus_rp2.pdf}
\end{frame}
